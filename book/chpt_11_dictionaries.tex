

\chapter{Dictionaries  |  字典}

This chapter presents another built-in type called a dictionary.
Dictionaries are one of Python's best features; they are the
building blocks of many efficient and elegant algorithms.

本章介绍另一个内建数据类型: 字典 (dictionary)。
字典是 Python 中最优秀的特性之一; 许多高效、优雅的算法以此为基础。

\section{A dictionary is a mapping  |  字典即映射}

\index{dictionary}  \index{dictionary}
\index{type!dict}  \index{key}
\index{key-value pair}  \index{index}

A {\bf dictionary} is like a list, but more general.  In a list,
the indices have to be integers; in a dictionary they can
be (almost) any type.

{\em 字典} 与列表类似,但是更加通用。
在列表中,索引必须是整数;但在字典中,它们可以是(几乎)任何类型。

A dictionary contains a collection of indices, which are called {\bf
  keys}, and a collection of values.  Each key is associated with a
single value.  The association of a key and a value is called a {\bf
  key-value pair} or sometimes an {\bf item}.  \index{item}

字典包含了一个索引的集合,被称为 {\em 键} (keys) ,和一个 {\em 值} (values)的集合。 一个键对应一个值。 这种一一对应的关联被称为 {\em 键值对} (key-value pair) , 有时也被称为 {\em 项} (item) 。

In mathematical language, a dictionary represents a {\bf mapping}
from keys to values, so you can also say that each key
``maps to'' a value.
As an example, we'll build a dictionary that maps from English
to Spanish words, so the keys and the values are all strings.

在数学语言中,字典表示的是从键到值的 {\em 映射},所以你也可以说每一个键 ``映射到'' 一个值。  举个例子,我们接下来创建一个字典,将英语单词映射至西班牙语单词,因此键和值都是字符串。

The function {\tt dict} creates a new dictionary with no items.
Because {\tt dict} is the name of a built-in function, you
should avoid using it as a variable name.

\li{dict} 函数生成一个不含任何项的新字典。 由于 \li{dict} 是内建函数名,你应该避免使用它来命名变量。

\index{dict function}  \index{function!dict}

\begin{lstlisting}
>>> eng2sp = dict()
>>> eng2sp
{}
\end{lstlisting}

The squiggly-brackets, \verb"{}", represent an empty dictionary.
To add items to the dictionary, you can use square brackets:

花括号 \verb"{}" 表示一个空字典。你可以使用方括号向字典中增加项:

\index{squiggly bracket}  \index{bracket!squiggly}

\begin{lstlisting}
>>> eng2sp['one'] = 'uno'
\end{lstlisting}

%
This line creates an item that maps from the key
\verb"'one'" to the value \verb"'uno'".  If we print the
dictionary again, we see a key-value pair with a colon
between the key and value:

这行代码创建一个新项,将键 \li{'one'} 映射至值 \li{'uno'}。
如果我们再次打印该字典,会看到一个以冒号分隔的键值对:

\begin{lstlisting}
>>> eng2sp
{'one': 'uno'}
\end{lstlisting}

%
This output format is also an input format.  For example,
you can create a new dictionary with three items:

输出的格式同样也是输入的格式。 例如,你可以像这样创建一个包含三个项的字典:

\begin{lstlisting}
>>> eng2sp = {'one': 'uno', 'two': 'dos', 'three': 'tres'}
\end{lstlisting}

%
But if you print {\tt eng2sp}, you might be surprised:

但是,如果你打印 \li{eng2sp} ,结果可能会让你感到意外:

\begin{lstlisting}
>>> eng2sp
{'one': 'uno', 'three': 'tres', 'two': 'dos'}
\end{lstlisting}

%
The order of the key-value pairs might not be the same.  If
you type the same example on your computer, you might get a
different result.  In general, the order of items in
a dictionary is unpredictable.

键-值对的顺序和原来不同。
同样的例子在你的电脑上可能有不同的结果。通常来说,字典中项的顺序是不可预知的。

But that's not a problem because
the elements of a dictionary are never indexed with integer indices.
Instead, you use the keys to look up the corresponding values:

但这没有关系,因为字典的元素不使用整数索引来索引,而是用键来查找对应的值:

\begin{lstlisting}
>>> eng2sp['two']
'dos'
\end{lstlisting}

%
The key \verb"'two'" always maps to the value \verb"'dos'" so the order
of the items doesn't matter.

键 \li{'two'} 总是映射到值 \li{'dos'} ,因此项的顺序没有关系。

If the key isn't in the dictionary, you get an exception:

如果键不存在字典中,会抛出一个异常:

\index{exception!KeyError}  \index{KeyError}

\begin{lstlisting}
>>> eng2sp['four']
KeyError: 'four'
\end{lstlisting}

%
The {\tt len} function works on dictionaries; it returns the
number of key-value pairs:

\li{len} 函数也适用于字典;它返回键值对的个数:

\index{len function}  \index{function!len}

\begin{lstlisting}
>>> len(eng2sp)
3
\end{lstlisting}

%
The {\tt in} operator works on dictionaries, too; it tells you whether
something appears as a {\em key} in the dictionary (appearing
as a value is not good enough).

\li{in} 操作符也适用于字典;它可以用来检验字典中是否存在某个 {\em 键} (仅仅有这个值还不够)。

\index{membership!dictionary}  \index{in operator}
\index{operator!in}

\begin{lstlisting}
>>> 'one' in eng2sp
True
>>> 'uno' in eng2sp
False
\end{lstlisting}

%
To see whether something appears as a value in a dictionary, you
can use the method {\tt values}, which returns a collection of
values, and then use the {\tt in} operator:

想要知道字典中是否存在某个值,你可以使用 \li{values} 方法,它返回值的集合,然后你可以使用 \li{in} 操作符来验证:

\index{values method}  \index{method!values}

\begin{lstlisting}
>>> vals = eng2sp.values()
>>> 'uno' in vals
True
\end{lstlisting}

%
The {\tt in} operator uses different algorithms for lists and
dictionaries.  For lists, it searches the elements of the list in
order, as in Section~\ref{find}.  As the list gets longer, the search
time gets longer in direct proportion.

\li{in} 操作符对列表和字典采用不同的算法。
对于列表,它按顺序依次查找目标,如 \ref{find}~节 所示。
随着列表的增长,搜索时间成正比增长。

For dictionaries, Python uses an
algorithm called a {\bf hashtable} that has a remarkable property: the
{\tt in} operator takes about the same amount of time no matter how
many items are in the dictionary.  I explain how that's possible
in Section~\ref{hashtable}, but the explanation might not make
sense until you've read a few more chapters.

对于字典,Python 使用一种叫做 {\em 哈希表} (hashtable) 的算法,
这种算法具备一种了不起的特性: 无论字典中有多少项, \li{in} 运算符搜索所需的时间都是一样的。 我将在第二十一章的哈希表一节中具体解释背后的原理,
但是如果你不再多学习几章内容,现在去看解释的话可能很难理解。

\section{Dictionary as a collection of counters  |  字典作为计数器集合}
\label{histogram}
\index{counter}

Suppose you are given a string and you want to count how many
times each letter appears.  There are several ways you could do it:

假设给你一个字符串,你想计算每个字母出现的次数。
有多种方法可以使用:

\begin{enumerate}

\item You could create 26 variables, one for each letter of the
alphabet.  Then you could traverse the string and, for each
character, increment the corresponding counter, probably using
a chained conditional.

\item You could create a list with 26 elements.  Then you could
convert each character to a number (using the built-in function
{\tt ord}), use the number as an index into the list, and increment
the appropriate counter.

\item You could create a dictionary with characters as keys
and counters as the corresponding values.  The first time you
see a character, you would add an item to the dictionary.  After
that you would increment the value of an existing item.

\end{enumerate}

\begin{enumerate}

\item 你可以生成26个变量,每个对应一个字母表中的字母。然后你可以遍历字符串,对于 每个字符,递增相应的计数器,你可能会用到链式条件。

\item 你可以生成具有26个元素的列表。 然后你可以将每个字符转化为一个数字(使用内建函数 \li{ord} ),使用这些数字作为列表的索引,并递增适当的计数器。

\item 你可以生成一个字典,将字符作为键,计数器作为相应的值。字母第一次出现时,你应该向字典中增加一项。 这之后,你应该递增一个已有项的值。

\end{enumerate}

Each of these options performs the same computation, but each
of them implements that computation in a different way.

每个方法都是为了做同一件事,但是各自的实现方法不同。

\index{implementation}

An {\bf implementation} is a way of performing a computation;
some implementations are better than others.  For example,
an advantage of the dictionary implementation is that we don't
have to know ahead of time which letters appear in the string
and we only have to make room for the letters that do appear.

{\em 实现} 是指执行某种计算的方法;有的实现更好。
例如,使用字典的实现有一个优势,即我们不需要事先知道字符串中有几种字母,
只要在出现新字母时分配空间就

Here is what the code might look like:

代码可能是这样的:

\begin{lstlisting}
def histogram(s):
    d = dict()
    for c in s:
        if c not in d:
            d[c] = 1
        else:
            d[c] += 1
    return d
\end{lstlisting}

%
The name of the function is {\tt histogram}, which is a statistical
term for a collection of counters (or frequencies).
函数名叫 \li{histogram} (直方图) ,是计数器(或是频率)集合的统计术语。

\index{histogram}  \index{frequency}
\index{traversal}

The first line of the
function creates an empty dictionary.  The {\tt for} loop traverses
the string.  Each time through the loop, if the character {\tt c} is
not in the dictionary, we create a new item with key {\tt c} and the
initial value 1 (since we have seen this letter once).  If {\tt c} is
already in the dictionary we increment {\tt d[c]}.

函数的第一行生成一个空字典。 \li{for} 循环遍历该字符串。
每次循环,如果字符 \li{c} 不在字典中, 我们用键 \li{c} 和初始值 \li{1} 生成一个新项 (因为该字母出现了一次)。 如果 \li{c} 已经在字典中了,那么我们递增 \li{d[c]} 。

\index{histogram}

Here's how it works:

下面是运行结果:

\begin{lstlisting}
>>> h = histogram('brontosaurus')
>>> h
{'a': 1, 'b': 1, 'o': 2, 'n': 1, 's': 2, 'r': 2, 'u': 2, 't': 1}
\end{lstlisting}

%
The histogram indicates that the letters \verb"'a'" and \verb"'b'"
appear once; \verb"'o'" appears twice, and so on.

\li{histogram} 函数表明字母 \li{'a'} 和 \li{'b'} 出现了一次,  \li{'o'} 出现了两次,等等。

\index{get method}  \index{method!get}

Dictionaries have a method called {\tt get} that takes a key
and a default value.  If the key appears in the dictionary,
{\tt get} returns the corresponding value; otherwise it returns
the default value.  For example:

字典类有一个 \li{get} 方法,接受一个键和一个默认值作为参数。
如果字典中存在该键,则返回对应值;否则返回传入的默认值。 例如:

\begin{lstlisting}
>>> h = histogram('a')
>>> h
{'a': 1}
>>> h.get('a', 0)
1
>>> h.get('b', 0)
0
\end{lstlisting}

%
As an exercise, use {\tt get} to write {\tt histogram} more concisely.  You
should be able to eliminate the {\tt if} statement.

我们做个练习,试着用 \li{get} 简化 \li{histogram} 函数。你应该能够不再使用 \li{if} 语句。


\section{Looping and dictionaries  |  循环和字典}

\index{dictionary!looping with}  \index{looping!with dictionaries}
\index{traversal}

If you use a dictionary in a {\tt for} statement, it traverses
the keys of the dictionary.  For example, \verb"print_hist"
prints each key and the corresponding value:

在 \li{for} 循环中使用字典会遍历其所有的键。
例如,下面的 \li{print_hist} 会打印所有键与对应的值:

\begin{lstlisting}
def print_hist(h):
    for c in h:
        print(c, h[c])
\end{lstlisting}

%
Here's what the output looks like:

输出类似:

\begin{lstlisting}
>>> h = histogram('parrot')
>>> print_hist(h)
a 1
p 1
r 2
t 1
o 1
\end{lstlisting}

%
Again, the keys are in no particular order.  To traverse the keys
in sorted order, you can use the built-in function {\tt sorted}:

重申一遍,字典中的键是无序的。
如果要以确定的顺序遍历字典,你可以使用内建方法 \li{sorted}:

\index{keys method}  \index{method!keys}

\begin{lstlisting}
>>> for key in sorted(h):
...     print(key, h[key])
a 1
o 1
p 1
r 2
t 1
\end{lstlisting}

%TODO: get this on Atlas


\section{Reverse lookup  |  逆向查找}
\label{raise}

\index{dictionary!lookup}  \index{dictionary!reverse lookup}
\index{lookup, dictionary}  \index{reverse lookup, dictionary}

Given a dictionary {\tt d} and a key {\tt k}, it is easy to
find the corresponding value {\tt v = d[k]}.  This operation
is called a {\bf lookup}.

给定一个字典 \li{d} 以及一个键 \li{t} ,很容易找到相应的值 \li{v = d[k]} 。
该运算被称作 {\em 查找} (lookup) 。

But what if you have {\tt v} and you want to find {\tt k}?
You have two problems: first, there might be more than one
key that maps to the value {\tt v}.  Depending on the application,
you might be able to pick one, or you might have to make
a list that contains all of them.  Second, there is no
simple syntax to do a {\bf reverse lookup}; you have to search.

但是如果你想通过 \li{v} 找到 \li{k} 呢?
有两个问题:第一,可能有不止一个的键其映射到值v。
你可能可以找到唯一一个,不然就得用 \li{list} 把所有的键包起来。
第二,没有简单的语法可以完成 {\em 逆向查找} (reverse lookup) ; 你必须搜索。

Here is a function that takes a value and returns the first
key that maps to that value:

下面这个函数接受一个值并返回映射到该值的第一个键:

\begin{lstlisting}
def reverse_lookup(d, v):
    for k in d:
        if d[k] == v:
            return k
    raise LookupError()
\end{lstlisting}

%
This function is yet another example of the search pattern, but it
uses a feature we haven't seen before, {\tt raise}.  The
{\bf raise statement} causes an exception; in this case it causes a
{\tt LookupError}, which is a built-in exception used to indicate
that a lookup operation failed.

该函数是搜索模式的另一个例子,但是它使用了一个我们之前没有见过的特性,\li{raise}。 \li{raise} 语句 能触发异常,这里它触发了 \li{ValueError},这是一个表示查找操作失败的内建异常。

\index{search}  \index{pattern!search}
\index{raise statement} \index{statement!raise}
\index{exception!LookupError} \index{LookupError}

If we get to the end of the loop, that means {\tt v}
doesn't appear in the dictionary as a value, so we raise an
exception.

如果我们到达循环结尾,这意味着字典中不存在 \li{v} 这个值,所以我们触发一个异常。

Here is an example of a successful reverse lookup:

下面是一个成功逆向查找的例子:

\begin{lstlisting}
>>> h = histogram('parrot')
>>> key = reverse_lookup(h, 2)
>>> key
'r'
\end{lstlisting}

%
And an unsuccessful one:

以及一个失败的例子:

\begin{lstlisting}
>>> key = reverse_lookup(h, 3)
Traceback (most recent call last):
  File "<stdin>", line 1, in <module>
  File "<stdin>", line 5, in reverse_lookup
LookupError
\end{lstlisting}

%
The effect when you raise an exception is the same as when
Python raises one: it prints a traceback and an error message.

你触发的异常和 Python 触发的产生效果一样:都打印一条回溯和错误信息。

\index{traceback}  \index{optional argument}
\index{argument!optional}

The {\tt raise} statement can take a detailed error message as an
optional argument.  For example:

\li{raise} 语句接受一个详细的错误信息作为可选的实参。  例如:

\begin{lstlisting}
>>> raise LookupError('value does not appear in the dictionary')
Traceback (most recent call last):
  File "<stdin>", line 1, in ?
LookupError: value does not appear in the dictionary
\end{lstlisting}

%
A reverse lookup is much slower than a forward lookup; if you
have to do it often, or if the dictionary gets big, the performance
of your program will suffer.

逆向查找比正向查找慢得多; 如果你频繁执行这个操作或是字典很大,程序性能会变差。

\section{Dictionaries and lists  |  字典和列表}
\label{invert}

Lists can appear as values in a dictionary.  For example, if you
are given a dictionary that maps from letters to frequencies, you
might want to invert it; that is, create a dictionary that maps
from frequencies to letters.  Since there might be several letters
with the same frequency, each value in the inverted dictionary
should be a list of letters.

在字典中,列表可以作为值出现。
例如,如果你有一个从字母映射到频率的字典, 而你想倒转它;
也就是生成一个从频率映射到字母的字典。
因为可能有些字母具有相同的频率,所以在倒转字典中的每个值应该是一个字母组成的列表。

\index{invert dictionary}  \index{dictionary!invert}

Here is a function that inverts a dictionary:

下面是一个倒转字典的函数:

\begin{lstlisting}
def invert_dict(d):
    inverse = dict()
    for key in d:
        val = d[key]
        if val not in inverse:
            inverse[val] = [key]
        else:
            inverse[val].append(key)
    return inverse
\end{lstlisting}

%
Each time through the loop, {\tt key} gets a key from {\tt d} and
{\tt val} gets the corresponding value.  If {\tt val} is not in {\tt
  inverse}, that means we haven't seen it before, so we create a new
item and initialize it with a {\bf singleton} (a list that contains a
single element).  Otherwise we have seen this value before, so we
append the corresponding key to the list.  \index{singleton}

每次循环, \li{key} 从 \li{d} 获得一个键和相应的值 \li{val} 。
如果 \li{val} 不在 \li{inverse} 中,意味着我们之前没有见过它,
因此我们生成一个新项并用一个 {\em 单元素集合} (singleton) (只包含一个元素的列表)初始化它。 否则就意味着之前已经见过该值,因此将其对应的键添加至列表。

Here is an example:

举个例子:

\begin{lstlisting}
>>> hist = histogram('parrot')
>>> hist
{'a': 1, 'p': 1, 'r': 2, 't': 1, 'o': 1}
>>> inverse = invert_dict(hist)
>>> inverse
{1: ['a', 'p', 't', 'o'], 2: ['r']}
\end{lstlisting}

\begin{figure}
\centerline
{\includegraphics[scale=0.8]{../source/figs/dict1.pdf}}
\caption{State diagram.}
\label{fig.dict1}
\end{figure}

Figure~\ref{fig.dict1} is a state diagram showing {\tt hist} and {\tt inverse}.
A dictionary is represented as a box with the type {\tt dict} above it
and the key-value pairs inside.  If the values are integers, floats or
strings, I draw them inside the box, but I usually draw lists
outside the box, just to keep the diagram simple.

图~\ref{fig.dict1} 是关于 \li{hist} 与 \li{inverse} 的状态图。字典用标有类型 \li{dict} 的方框表示,方框中是键值对。如果值是整数、浮点数或字符串,
我就把它们画在方框内部,但我通常把列表画在方框外面,目的只是为了不让图表变复杂。

\index{state diagram}  \index{diagram!state}

Lists can be values in a dictionary, as this example shows, but they
cannot be keys.  Here's what happens if you try:

如本例所示,列表可以作为字典中的值,但是不能是键。
下面演示了这样做的结果:

\index{TypeError}  \index{exception!TypeError}

\begin{lstlisting}
>>> t = [1, 2, 3]
>>> d = dict()
>>> d[t] = 'oops'
Traceback (most recent call last):
  File "<stdin>", line 1, in ?
TypeError: list objects are unhashable
\end{lstlisting}

%
I mentioned earlier that a dictionary is implemented using
a hashtable and that means that the keys have to be {\bf hashable}.

我之前提过,字典使用哈希表实现,这意味着键必须是 {\bf 可哈希的} (hashable) 。

\index{hash function}  \index{hashable}

A {\bf hash} is a function that takes a value (of any kind)
and returns an integer.  Dictionaries use these integers,
called hash values, to store and look up key-value pairs.


{\em 哈希} (hash) 函数接受一个值 (任何类型) 并返回一个整数。
字典使用被称作哈希值的这些整数,来存储和查找键值对。
\index{immutability}

This system works fine if the keys are immutable.  But if the
keys are mutable, like lists, bad things happen.  For example,
when you create a key-value pair, Python hashes the key and
stores it in the corresponding location.  If you modify the
key and then hash it again, it would go to a different location.
In that case you might have two entries for the same key,
or you might not be able to find a key.  Either way, the
dictionary wouldn't work correctly.

如果键是不可变的,那么这种实现可以很好地工作。
但是如果键是可变的,如列表,那么就会发生糟糕的事情。
例如,当你生成一个键值对时,Python哈希该键并将其存储在相应的位置。
如果你改变键然后再次哈希它,它将被存储到另一个位置。
在那种情况下,对于相同的键,你可能有两个值, 或者你可能无法找到一个键。
无论如何,字典都不会正确的工作。

That's why keys have to be hashable, and why mutable types like
lists aren't.  The simplest way to get around this limitation is to
use tuples, which we will see in the next chapter.

这就是为什么键必须是可哈希的,以及为什么如列表这种可变类型不能作为键。
绕过这种限制最简单的方法是使用元组, 我们将在下一章中介绍。

Since dictionaries are mutable, they can't be used as keys,
but they {\em can} be used as values.

因为字典是可变的,因此它们不能作为键,但是 {\em 可以} 用作值。


\section{Memos  |  备忘}
\label{memoize}

If you played with the {\tt fibonacci} function from
Section~\ref{one.more.example}, you might have noticed that the bigger
the argument you provide, the longer the function takes to run.
Furthermore, the run time increases quickly.

如果你在 \ref{one.more.example}节中接触过 \li{fibonacci} 函数, 你可能注意到输入的实参越大,函数运行就需要越多时间。
而且运行时间增长得非常快。

\index{fibonacci function}  \index{function!fibonacci}

To understand why, consider Figure~\ref{fig.fibonacci}, which shows
the {\bf call graph} for {\tt fibonacci} with {\tt n=4}:

要理解其原因,思考 图~\ref{fig.fibonacci} ,它展示了当 \li{n=4} 时 \li{fibonacci} 的 {\em 调用图} (call graph) :

\begin{figure}
\centerline
{\includegraphics[scale=0.7]{../source/figs/fibonacci.pdf}}
\caption{Call graph.}
\label{fig.fibonacci}
\end{figure}

A call graph shows a set of function frames, with lines connecting each
frame to the frames of the functions it calls.  At the top of the
graph, {\tt fibonacci} with {\tt n=4} calls {\tt fibonacci} with {\tt
n=3} and {\tt n=2}.  In turn, {\tt fibonacci} with {\tt n=3} calls
{\tt fibonacci} with {\tt n=2} and {\tt n=1}.  And so on.

调用图中列出了一系列函数栈帧,每个栈帧之间通过线条与调用它的函数栈帧相连。
在图的顶端,\li{n = 4} 的 \li{fibonacci} 调用 \li{n = 3} 和 \li{n = 2} 的 \li{fibonacci} 。 接着, \li{n = 3} 的 \li{fibonacci} 调用 \li{n = 2} 和 \li{n = 1} 的 \li{fibonacci}。 以此类推。

\index{function frame}  \index{frame}
\index{call graph}

Count how many times {\tt fibonacci(0)} and {\tt fibonacci(1)} are
called.  This is an inefficient solution to the problem, and it gets
worse as the argument gets bigger.

数数 \li{fibonacci(0)} 和 \li{fibonacci(1)} 总共被调用了几次。
对该问题,这不是一个高效的解,并且随着实参的变大会变得更糟。

\index{memo}

One solution is to keep track of values that have already been
computed by storing them in a dictionary.  A previously computed value
that is stored for later use is called a {\bf memo}.  Here is a
``memoized'' version of {\tt fibonacci}:

一个解决办法是保存已经计算过的值,将它们存在一个字典中。
存储之前计算过的值以便今后使用,它被称作 {\em 备忘录} (memo) 。
下面是使用备忘录 (memoized) 的 \li{fibonacci} 的实现:

\begin{lstlisting}
known = {0:0, 1:1}

def fibonacci(n):
    if n in known:
        return known[n]

    res = fibonacci(n-1) + fibonacci(n-2)
    known[n] = res
    return res
\end{lstlisting}

%
{\tt known} is a dictionary that keeps track of the Fibonacci
numbers we already know.  It starts with
two items: 0 maps to 0 and 1 maps to 1.

\li{known} 是一个字典,记录了我们已经计算过的斐波纳契数字。
它一开始包含两个项:0映射到0,1映射到1。

Whenever {\tt fibonacci} is called, it checks {\tt known}.
If the result is already there, it can return
immediately.  Otherwise it has to
compute the new value, add it to the dictionary, and return it.

当 \li{fibonacci} 被调用时,它先检查 \li{known} 。 如果结果存在,则立即返回。 否则,它必须计算新的值,将其加入字典,并返回它。

If you run this version of {\tt fibonacci} and compare it with
the original, you will find that it is much faster.

将两个版本的 \li{fibonacci} 函数比比看,你就知道后者快了很多。


\section{Global variables  |  全局变量}

\index{global variable}  \index{variable!global}

In the previous example, {\tt known} is created outside the function,
so it belongs to the special frame called \verb"__main__".
Variables in \verb"__main__" are sometimes called {\bf global}
because they can be accessed from any function.  Unlike local
variables, which disappear when their function ends, global variables
persist from one function call to the next.

在前面的例子中,\li{known} 是在函数的外部创建的,
因此它属于被称作 \li{__main__} 的特殊帧。
因为 \li{__main__} 中的变量可以被任何函数访问,它们也被称作 {\em 全局变量} (global) 。 与函数结束时就会消失的局部变量不同,不同函数调用时全局变量一直都存在。

\index{flag}

It is common to use global variables for {\bf flags}; that is,
boolean variables that indicate (``flag'') whether a condition
is true.  For example, some programs use
a flag named {\tt verbose} to control the level of detail in the
output:

全局变量普遍用作 {\em 标记} (flag); 就是说明(标记)一个条件是否为真的布尔变量。
例如,一些程序使用一个被称作 \li{verbose} 的标记来控制输出的丰富程度:

\begin{lstlisting}
verbose = True

def example1():
    if verbose:
        print('Running example1')
\end{lstlisting}

%
If you try to reassign a global variable, you might be surprised.
The following example is supposed to keep track of whether the
function has been called:

如果你试图对一个全局变量重新赋值,结果可能出乎意料。
下面的例子本应该记录函数是否已经被调用过了

\index{reassignment}

\begin{lstlisting}
been_called = False

def example2():
    been_called = True         # WRONG
\end{lstlisting}

%
But if you run it you will see that the value of \verb"been_called"
doesn't change.  The problem is that {\tt example2} creates a new local
variable named \verb"been_called".  The local variable goes away when
the function ends, and has no effect on the global variable.

但是如果你运行它,你会发现 \li{been_called} 的值并未发生改变。
问题在于 \li{example2} 生成了一个新的被称作 \li{been_called} 的局部变量。
当函数结束的时候,该局部变量也消失了,并且对全局变量没有影响。

\index{global statement}  \index{statement!global}
\index{declaration}

To reassign a global variable inside a function you have to
{\bf declare} the global variable before you use it:

要在函数内对全局变量重新赋值,你必须在使用之前 {\em 声明} (declare) 该全局变量:

\begin{lstlisting}
been_called = False

def example2():
    global been_called
    been_called = True
\end{lstlisting}

%
The {\bf global statement} tells the interpreter
something like, ``In this function, when I say \verb"been_called", I
mean the global variable; don't create a local one.''

\li{global} 语句 告诉编译器,``在这个函数里,当我说 \li{been_called} 时,我指的是那个全局变量,别生成局部变量''。

\index{update!global variable}  \index{global variable!update}

Here's an example that tries to update a global variable:

下面是一个试图更新全局变量的例子:

\begin{lstlisting}
count = 0

def example3():
    count = count + 1          # WRONG
\end{lstlisting}

%
If you run it you get:

一旦运行,你会发现:

\index{UnboundLocalError}  \index{exception!UnboundLocalError}

\begin{lstlisting}
UnboundLocalError: local variable 'count' referenced before assignment
\end{lstlisting}

%
Python assumes that {\tt count} is local, and under that assumption
you are reading it before writing it.  The solution, again,
is to declare {\tt count} global.

Python默认 \li{count} 是局部变量,在这个假设下,你这是在未写入任何东西前就试图读取。
解决方法还是声明 \li{count} 是全局变量。

\index{counter}

\begin{lstlisting}
def example3():
    global count
    count += 1
\end{lstlisting}

%
If a global variable refers to a mutable value, you can modify
the value without declaring the variable:

如果全局变量是可变的,你可以不加声明地修改它:

\index{mutability}

\begin{lstlisting}
known = {0:0, 1:1}

def example4():
    known[2] = 1
\end{lstlisting}

%
So you can add, remove and replace elements of a global list or
dictionary, but if you want to reassign the variable, you
have to declare it:

因此你可以增加、删除和替代全局列表或者字典的元素,
但是如果你想对变量重新赋值,你必须声明它:

\begin{lstlisting}
def example5():
    global known
    known = dict()
\end{lstlisting}

%
Global variables can be useful, but if you have a lot of them,
and you modify them frequently, they can make programs
hard to debug.

全局变量有时是很有用的,但如果你的程序中有很多全局变量,而且修改频繁,
这样会增加程序调试的难度。

\section{Debugging  |  调试}
\index{debugging}

As you work with bigger datasets it can become unwieldy to
debug by printing and checking the output by hand.  Here are some
suggestions for debugging large datasets:

当你操作较大的数据集时,通过打印并手工检查数据来调试很不方便。
下面是针对调试大数据集的一些建议:

\begin{description}

\item[Scale down the input:] If possible, reduce the size of the
dataset.  For example if the program reads a text file, start with
just the first 10 lines, or with the smallest example you can find.
You can either edit the files themselves, or (better) modify the
program so it reads only the first {\tt n} lines.

If there is an error, you can reduce {\tt n} to the smallest
value that manifests the error, and then increase it gradually
as you find and correct errors.

\item[Check summaries and types:] Instead of printing and checking the
entire dataset, consider printing summaries of the data: for example,
the number of items in a dictionary or the total of a list of numbers.

A common cause of runtime errors is a value that is not the right
type.  For debugging this kind of error, it is often enough to print
the type of a value.

\item[Write self-checks:]  Sometimes you can write code to check
for errors automatically.  For example, if you are computing the
average of a list of numbers, you could check that the result is
not greater than the largest element in the list or less than
the smallest.  This is called a ``sanity check'' because it detects
results that are ``insane''.

\index{sanity check}  \index{consistency check}

Another kind of check compares the results of two different
computations to see if they are consistent.  This is called a
``consistency check''.

\item[Format the output:] Formatting debugging output
can make it easier to spot an error.  We saw an example in
Section~\ref{factdebug}.  The {\tt pprint} module provides
a {\tt pprint} function that displays built-in types in
a more human-readable format ({\tt pprint} stands for
``pretty print'').

\index{pretty print}  \index{pprint module}
\index{module!pprint}

\end{description}

\begin{description}

\item[缩小输入:] 如果可能,减小数据集合的大小。
    例如,如果程序读入一个文本文件,从前10行开始分析,或是找到更小的样例。
    你可以选择编辑读入的文件,或是(最好)修改程序使它只读入前 \li{n} 行。

    如果出错了,你可以将 \li{n} 缩小为会导致该错误的最小值,然后在查找和解决错误的同时,逐步增加 n 的值。

\item[检查摘要和类型:] Instea考虑打印数据的摘要,而不是打印并检查全部数据集合:
    例如,字典中项的数目或者数字列表的总和。

    运行时错误的一个常见原因,是值的类型不正确。 为了调试此类错误,打印值的类型通常就足够了。

\item[编写自检代码:]  有时你可以写代码来自动检查错误。 例如,如果你正在计算数字列表的平均数,你可以检查其结果是不是大于列表中最大的元素,或者小于最小的元素。 这被称 作``合理性检查'',因为它能检测出``不合理的''结果。

\index{sanity check}  \index{consistency check}

另一类检查是比较两个不同计算的结果,来看一下它们是否一致。这被称作 ``一致性检查''。

\item[格式化输出:] Formatting debugging output
can make it easier to spot an error.  We saw an example in
Section~\ref{factdebug}.  The {\tt pprint} module provides
a {\tt pprint} function that displays built-in types in
a more human-readable format ({\tt pprint} stands for
``pretty print'').

格式化调试输出能够更容易定位一个错误。 我们在 \ref{factdebug} 一节中看过一个示例。 \li{pprint} 模块提供了一个 \li{pprint} 函数,它可以更可读的格式显示内建类型( \li{pprint} 代表 ``pretty print'')。

\index{pretty print}  \index{pprint module}
\index{module!pprint}

\end{description}

Again, time you spend building scaffolding can reduce
the time you spend debugging.

重申一次,你花在搭建脚手架上的时间能减少你花在调试上的时间。

\index{scaffolding}


\section{Glossary  |  术语表}

\begin{description}

\item[mapping:] A relationship in which each element of one set
corresponds to an element of another set.

\index{mapping}

\item[映射(mapping):] 一个集合中的每个元素对应另一个集合中的一个元素的关系。

\index{mapping}

\item[dictionary:] A mapping from keys to their
corresponding values.

\index{dictionary}

\item[字典(dictionary):] 将键映射到对应值的映射。

\index{dictionary}

\item[key-value pair:] The representation of the mapping from
a key to a value.

\index{key-value pair}

\item[键值对(key-value pair):] 键值之间映射关系的呈现形式。

\index{key-value pair}

\item[item:] In a dictionary, another name for a key-value
  pair.

\index{item!dictionary}

\item[项(item):] 在字典中,这是键值对的另一个名称。

\index{item!dictionary}

\item[key:] An object that appears in a dictionary as the
first part of a key-value pair.

\index{key}

\item[键(key):] 字典中作为键值对第一部分的对象。

\index{key}

\item[value:] An object that appears in a dictionary as the
second part of a key-value pair.  This is more specific than
our previous use of the word ``value''.

\index{value}

\item[值(value):] 字典中作为键值对第二部分的对象。它比我们之前所用的``值''一词更具体。

\index{value}

\item[implementation:] A way of performing a computation.

\index{implementation}

\item[实现(implementation):] 执行计算的一种形式。

\index{implementation}

\item[hashtable:] The algorithm used to implement Python
dictionaries.

\index{hashtable}

\item[哈希表(hashtable):] 用来实现Python字典的算法。

\index{hashtable}

\item[hash function:] A function used by a hashtable to compute the
location for a key.

\index{hash function}

\item[哈希函数(hash function):] 哈希表用来计算键的位置的函数。

\index{hash function}

\item[hashable:] A type that has a hash function.  Immutable
types like integers,
floats and strings are hashable; mutable types like lists and
dictionaries are not.

\index{hashable}

\item[可哈希的(hashable):] 具备哈希函数的类型。诸如整数、浮点数和字符串这样的不可变类型是可哈希的;诸如列表和字典这样的可变对象是不可哈希的。

\index{hashable}

\item[lookup:] A dictionary operation that takes a key and finds
the corresponding value.

\index{lookup}

\item[查找(lookup):] 接受一个键并返回相应值的字典操作。

\index{lookup}

\item[reverse lookup:] A dictionary operation that takes a value and finds
one or more keys that map to it.

\index{reverse lookup}

\item[逆向查找(reverse lookup):] 接受一个值并返回一个或多个映射至该值的键的字典操作。

\index{reverse lookup}

\item[raise statement:]  A statement that (deliberately) raises an exception.

\index{raise statement}  \index{statement!raise}

\item[raise语句:]  专门印发异常的一个语句。

\index{raise statement}  \index{statement!raise}

\item[singleton:] A list (or other sequence) with a single element.
\index{singleton}

\item[单元素集合(singleton):] 只有一个元素的列表(或其他序列)。
\index{singleton}

\item[call graph:] A diagram that shows every frame created during
the execution of a program, with an arrow from each caller to
each callee.

\index{call graph}  \index{diagram!call graph}

\item[调用图(call graph):] 绘出程序执行过程中创建的每个栈帧的调用图,其中的箭头从调用者指向被调用者。

\index{call graph}  \index{diagram!call graph}

\item[memo:] A computed value stored to avoid unnecessary future
computation.

\index{memo}

\item[备忘录(memo):] 一个存储的计算值,避免之后进行不必要的计算。

\index{memo}

\item[global variable:]  A variable defined outside a function.  Global
variables can be accessed from any function.

\index{global variable}

\item[全局变量(global variable):]  在函数外部定义的变量。任何函数都可以访问全局变量。

\index{global variable}

\item[global statement:]  A statement that declares a variable name
global.

\index{global statement}  \index{statement!global}

\item[global语句:]  将变量名声明为全局变量的语句。

\index{global statement}  \index{statement!global}

\item[flag:] A boolean variable used to indicate whether a condition
is true.

\index{flag}

\item[标记(flag):] 用于说明一个条件是否为真的布尔变量。

\index{flag}

\item[declaration:] A statement like {\tt global} that tells the
interpreter something about a variable.

\index{declaration}

\item[声明(declaration):] 类似 \li{global} 这种告知解释器如何处理变量的语句。

\index{declaration}

\end{description}



\section{Exercises  |  练习}

\begin{exercise}
\label{wordlist2}

\index{set membership}  \index{membership!set}

Write a function that reads the words in {\tt words.txt} and
stores them as keys in a dictionary.  It doesn't matter what the
values are.  Then you can use the {\tt in} operator
as a fast way to check whether a string is in
the dictionary.

If you did Exercise~\ref{wordlist1}, you can compare the speed
of this implementation with the list {\tt in} operator and the
bisection search.

编写一函数,读取 \li{words.txt} 中的单词并存储为字典中的键。值是什么无所谓。
然后,你可以使用 \li{in} 操作符检查一个字符串是否在字典中。

如果你做过 练习~\ref{wordlist1} ,可以比较一下 \li{in} 操作符和二分查找的速度。

\end{exercise}


\begin{exercise}
\label{setdefault}

Read the documentation of the dictionary method {\tt setdefault}
and use it to write a more concise version of \verb"invert_dict".
Solution: \url{http://thinkpython2.com/code/invert_dict.py}.

查看字典方法 \li{setdefault} 的文档,并使用该方法写一个更简洁的 \li{invert_dict}。

\index{setdefault method}  \index{method!setdefault}

\end{exercise}


\begin{exercise}
Memoize the Ackermann function from Exercise~\ref{ackermann} and see if
memoization makes it possible to evaluate the function with bigger
arguments.  Hint: no.
Solution: \url{http://thinkpython2.com/code/ackermann_memo.py}.

将 练习~\ref{ackermann} 中的 Ackermann 函数备忘录化 (memoize),看看备忘录化(memoization)是否可以支持解决更大的参数。 没有提示!

\href{http://thinkpython2.com/code/ackermann_memo.py}{参考答案}
\index{Ackermann function}  \index{function!ack}

\end{exercise}

\begin{exercise}

\index{duplicate}

If you did Exercise~\ref{duplicate}, you already have
a function named \verb"has_duplicates" that takes a list
as a parameter and returns {\tt True} if there is any object
that appears more than once in the list.

Use a dictionary to write a faster, simpler version of
\verb"has_duplicates".
Solution: \url{http://thinkpython2.com/code/has_duplicates.py}.

如果你做了 练习~\ref{duplicate} ,你就已经有了一个叫 \li{has_duplicates} 的函数,它接受一个列表作为参数,如果其中有某个对象在列表中出现不止一次就返回 \li{True}。

用字典写个更快、更简单的版本。

\href{http://thinkpython2.com/code/has_duplicates.py}{参考答案}


\end{exercise}


\begin{exercise}
\label{exrotatepairs}

\index{letter rotation}  \index{rotation!letters}

Two words are ``rotate pairs'' if you can rotate one of them
and get the other (see \verb"rotate_word" in Exercise~\ref{exrotate}).

Write a program that reads a wordlist and finds all the rotate
pairs.  Solution: \url{http://thinkpython2.com/code/rotate_pairs.py}.

两个单词如果反转其中一个就会得到另一个,则被称作 ``反转对''(参见 练习~\ref{exrotate} 中的 \li{rotate_word} 。

编写一程序,读入单词表并找到所有反转对。

\href{http://thinkpython2.com/code/rotate_pairs.py}{参考答案}

\end{exercise}


\begin{exercise}

\index{Car Talk}  \index{Puzzler}

Here's another Puzzler from {\em Car Talk}
(\url{http://www.cartalk.com/content/puzzlers}):

下面是取自 {\em Car Talk} 的另一个字谜题(http://www.cartalk.com/content/puzzlers):

\begin{quote}
This was sent in by a fellow named Dan O'Leary. He came upon a common
one-syllable, five-letter word recently that has the following unique
property. When you remove the first letter, the remaining letters form
a homophone of the original word, that is a word that sounds exactly
the same. Replace the first letter, that is, put it back and remove
the second letter and the result is yet another homophone of the
original word. And the question is, what's the word?

Now I'm going to give you an example that doesn't work. Let's look at
the five-letter word, `wrack.' W-R-A-C-K, you know like to `wrack with
pain.' If I remove the first letter, I am left with a four-letter
word, 'R-A-C-K.' As in, `Holy cow, did you see the rack on that buck!
It must have been a nine-pointer!' It's a perfect homophone. If you
put the `w' back, and remove the `r,' instead, you're left with the
word, `wack,' which is a real word, it's just not a homophone of the
other two words.

But there is, however, at least one word that Dan and we know of,
which will yield two homophones if you remove either of the first two
letters to make two, new four-letter words. The question is, what's
the word?
\end{quote}

\begin{quote}

这是来自一位名叫Dan O'Leary的朋友的分享。他有一次碰到了一个常见的单音节、有五个字母的单词,它具备以下独特的特性。当你移除第一个字母时,剩下的字母组成了原单词的同音词,即发音完全相同的单词。将第一个字母放回,然后取出第二个字母,结果又是原单词的另一个同音词。那么问题来了,这个单词是什么?

接下来我给大家举一个不满足要求的例子。我们来看一个五个字母的单词``wrack''。W-R-A-C-K,常用短句为 ``wrack with pain''。 如果我移除第一个字母,就剩下了一个四个字母的单词 ``R-A-C-K''。可以这么用,``Holy cow, did you see the rack on that buck! It must have been a nine-pointer!'' 它是一个完美的同音词。如果你把 ``w'' 放回去,移除 ``r'', 你得到的单词是 ``wack'' 。 这是一个真实的单词,但并不是前两个单词的同音词。

不过,我们和 Dan 知道至少有一个单词是满足这个条件的,即移除前两个字母中的任意一个,将会得到两个新的由四个字母组成的单词,而且发音完全一致。那么这个单词是什么呢?

\end{quote}

\index{homophone}  \index{reducible word}
\index{word, reducible}

You can use the dictionary from Exercise~\ref{wordlist2} to check
whether a string is in the word list.

你可以使用 练习~\ref{wordlist2} 中的字典检查某字符串是否出现在单词表中。

To check whether two words are homophones, you can use the CMU
Pronouncing Dictionary.  You can download it from
\url{http://www.speech.cs.cmu.edu/cgi-bin/cmudict} or from
\url{http://thinkpython2.com/code/c06d} and you can also download
\url{http://thinkpython2.com/code/pronounce.py}, which provides a function
named \verb"read_dictionary" that reads the pronouncing dictionary and
returns a Python dictionary that maps from each word to a string that
describes its primary pronunciation.

你可以使用CMU发音字典检查两个单词是否为同音词。从 \href{http://www.speech.cs.cmu.edu/cgi-bin/cmudict}{这里} 或 \href{http://thinkpython2.com/code/c06d}{这里} 下载。 你还可以下载\href{http://thinkpython2.com/code/pronounce.py}{ 这个脚本},其中提供了一个名叫 \li{read_dictionary} 的函数,可以读取发音字典,并返回一个将每个单词映射至描述其主要梵音的字符串的Python字典。

Write a program that lists all the words that solve the Puzzler.
Solution: \url{http://thinkpython2.com/code/homophone.py}.

编写一个程序,找到满足字谜题条件的所有单词。

\href{http://thinkpython2.com/code/homophone.py}{参考答案}

\end{exercise}
