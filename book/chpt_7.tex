

\chapter{Iteration  |  迭代}

This chapter is about iteration, which is the ability to run
a block of statements repeatedly.  We saw a kind of iteration,
using recursion, in Section~\ref{recursion}.
We saw another kind, using a {\tt for} loop,
in Section~\ref{repetition}.  In this chapter we'll see yet another
kind, using a {\tt while} statement.
But first I want to say a little more about variable assignment.

本章介绍迭代,即重复运行某个代码块的能力。我们已经在~\ref{recursion} 节接触了一种利用递归进行迭代的方式;在~\ref{repetition} 节中,接触了另一种利用 \li{for} 循环进行迭代的方式。 在本章中,我们将讨论另外一种利用 \li{while} 语句实现迭代的方式。
不过,首先我想再多谈谈有关变量赋值的问题。


\section{Reassignment  |  重新赋值}
\index{assignment}  \index{statement!assignment}  \index{reassignment}
\index{赋值}  \index{语句!赋值}  \index{重新赋值}

As you may have discovered, it is legal to make more than one
assignment to the same variable.  A new assignment makes an existing
variable refer to a new value (and stop referring to the old value).

可能你已发现对同一变量进行多次赋值是合法的。 新的赋值会使得已有的变量指向
新的值(同时不再指向旧的值)。

\begin{lstlisting}
>>> x = 5
>>> x
5
>>> x = 7
>>> x
7
\end{lstlisting}

%
The first time we display
{\tt x}, its value is 5; the second time, its
value is 7.

第一次打印 \li{x} 时, 它的值为 \li{5};第二次打印时,它的值是 \li{7}。

Figure~\ref{fig.assign2} shows what {\bf reassignment} looks
like in a state diagram.
\index{state diagram} \index{diagram!state}

图~\ref{fig.assign2} 展示了 {\em 重新赋值} 在状态图中看起来是什么样子。
\index{state diagram} \index{diagram!state}

At this point I want to address a common source of confusion.
Because Python uses the equal sign ({\tt =}) for assignment, it is
tempting to interpret a statement like {\tt a = b} as a mathematical
proposition of equality; that is, the claim that {\tt a} and
{\tt b} are equal.  But this interpretation is wrong.
\index{equality and assignment}

这里我想探讨一个常见的疑惑点。由于 Python 用等号 (\li{=}) 来赋值,所以很容易将 \li{a = b} 这样的语句理解为数学上的相等命题;即 \li{a} 和 \li{b} 相等。但是这种理解是错误的。

First, equality is a symmetric relationship and assignment is not.  For
example, in mathematics, if $a=7$ then $7=a$.  But in Python, the
statement {\tt a = 7} is legal and {\tt 7 = a} is not.

首先,相等是一种对称关系,赋值不是。例如,在数学上,如果 $a = 7$,
则 $7 = a$。 但是在 Python 中,语句 \li{a = 7} 是合法的, \li{7 = a} 则不合法。

Also, in mathematics, a proposition of equality is either true or
false for all time.  If $a=b$ now, then $a$ will always equal $b$.
In Python, an assignment statement can make two variables equal, but
they don't have to stay that way:

此外,数学中,相等命题不是对的就是错的。 如果 $a = b$,那么 $a$
则是永远与 $b$ 相等。在 Python 中,赋值语句可以使得两个变量相等,
但是这两个变量不一定必须保持这个状态:


\begin{lstlisting}
>>> a = 5
>>> b = a    # a and b are now equal
>>> a = 3    # a and b are no longer equal
>>> b
5
\end{lstlisting}

%
The third line changes the value of {\tt a} but does not change the
value of {\tt b}, so they are no longer equal.

第三行改变了 \li{a} 的值,但是没有改变 \li{b} 的值,所以它们不再相等了。

Reassigning variables is often useful, but you should use it
with caution.  If the values of variables change frequently, it can
make the code difficult to read and debug.

给变量重新赋值非常有用,但是需要小心使用。 对变量频繁重新赋值会使代码难于阅读,
不易调试。

\begin{figure}
\centerline
{\includegraphics[scale=0.8]{../source/figs/assign2.pdf}}
% \caption {State diagram.}
\caption {重新赋值的状态图。}
\label{fig.assign2}
\end{figure}



\section{Updating variables  |  更新变量}
\label{update}

\index{update}  \index{variable!updating}
\index{更新}  \index{变量!更新}

A common kind of reassignment is an {\bf update},
where the new value of the variable depends on the old.

重新赋值的一个常见方式是 {\em 更新} (update) , 更新操作中变量的新值会取决于旧值。


\begin{lstlisting}
>>> x = x + 1
\end{lstlisting}

%
This means ``get the current value of {\tt x}, add one, and then
update {\tt x} with the new value.''

这个语句的意思是,``获得 \li{x} 的当前值并与 \li{1} 做加法求和,然后将 \li{x} 的值更新为所求的和。''

If you try to update a variable that doesn't exist, you get an
error, because Python evaluates the right side before it assigns
a value to {\tt x}:

如果试图去更新一个不存在的变量,则会返回一个错误。 这是因为 Python 是先求
式子右边的值,然后再把所求的值赋给 \li{x}:

\begin{lstlisting}
>>> x = x + 1
NameError: name 'x' is not defined
\end{lstlisting}

%
Before you can update a variable, you have to {\bf initialize}
it, usually with a simple assignment:
\index{initialization (before update)}

在更新变量之前,你得先 {\em 初始化} (initialize) 它,通常是通过一个简单的赋值实现:
\index{initialization (before update)}

\begin{lstlisting}
>>> x = 0
>>> x = x + 1
\end{lstlisting}

%
Updating a variable by adding 1 is called an {\bf increment};
subtracting 1 is called a {\bf decrement}.
\index{increment}  \index{decrement}

通过加 \li{1} 来更新变量叫做 {\em 递增} (increment); 减 \li{1} 叫做 {\em 递减} (decrement)。
\index{increment}  \index{decrement}


\section{The {\tt while} statement  |  {\tt while} 语句}
\index{statement!while}  \index{while loop}
\index{loop!while}  \index{iteration}

Computers are often used to automate repetitive tasks.  Repeating
identical or similar tasks without making errors is something that
computers do well and people do poorly.  In a computer program,
repetition is also called {\bf iteration}.

计算机经常被用来自动处理重复性的任务。 计算机很擅长无纰漏地重复相同或者相似的任务, 而人类在这方面做的不好。 在计算机程序中, 重复也被称为 {\em 迭代} (iteration)。

We have already seen two functions, {\tt countdown} and
\verb"print_n", that iterate using recursion.  Because iteration is so
common, Python provides language features to make it easier.
One is the {\tt for} statement we saw in Section~\ref{repetition}.
We'll get back to that later.

我们已经见过两个利用递归来迭代的函数: \li{countdown} 和 \li{print_n} 。 由于迭代的使用非常普遍, 所以 Python 提供了使其更容易实现的语言特性。 其中之一就是我们在~\:ref{repetition} 一节看到的 \li{for} 语句。 后面我们还会继续介绍。

% add hyperref of 后面 here

Another is the {\tt while} statement.  Here is a version of {\tt
countdown} that uses a {\tt while} statement:

另外一个用于迭代的语句是 \li{while} 。 下面是使用 \li{while} 语句实现的 \li{countdown}:

\begin{lstlisting}
def countdown(n):
    while n > 0:
        print(n)
        n = n - 1
    print('Blastoff!')
\end{lstlisting}

%
You can almost read the {\tt while} statement as if it were English.
It means, ``While {\tt n} is greater than 0,
display the value of {\tt n} and then decrement
{\tt n}.  When you get to 0, display the word {\tt Blastoff!}''
\index{flow of execution}

你可以像读英语句子一样来读 \li{while} 语句。 它的意思是:``只要 \li{n} 的值大于 \li{0}, 则打印出 \li{n} 的值,然后让 \li{n} 减 \li{1}。 当 \li{n} 递减至 \li{0} 时,打印单词 \li{Blastoff}!''。

More formally, here is the flow of execution for a {\tt while} statement:

更正式地来说,\li{while} 语句的执行流程如下:

\begin{enumerate}

\item Determine whether the condition is true or false.

\item If false, exit the {\tt while} statement
and continue execution at the next statement.

\item If the condition is true, run the
body and then go back to step 1.

\end{enumerate}

\begin{enumerate}

\item 首先判断条件为 {\bf 真} 还是为 {\bf 假}。

\item 如果为假,退出 \li{while} 语句,然后执行接下来的语句;

\item 如果条件为真,则运行 \li{while} 语句 {\em 循环主体},运行完再返回第一步;

\end{enumerate}

This type of flow is called a loop because the third step
loops back around to the top.
\index{condition}  \index{loop}  \index{body}

这种形式的流程叫做 {\em 循环} (loop), 因为第三步后又循环回到了第一步。
\index{condition}  \index{loop}  \index{body}

The body of the loop should change the value of one or more variables
so that the condition becomes false eventually and the loop
terminates.  Otherwise the loop will repeat forever, which is called
an {\bf infinite loop}.  An endless source of amusement for computer
scientists is the observation that the directions on shampoo,
``Lather, rinse, repeat'', are an infinite loop.
\index{infinite loop}  \index{loop!infinite}

循环主体应该改变一个或多个变量的值,这样的话才能让条件判断最终变为假,
从而终止循环。 否则,循环将会永远重复下去,这被称为 {\em 无限循环} (infinite loop)。 在计算机科学家看来,洗发水的使用说明 —— ``抹洗发水,
清洗掉,重复'' 便是个无限循环,这总是会让他们觉得好笑。
\index{infinite loop}  \index{loop!infinite}

In the case of {\tt countdown}, we can prove that the loop
terminates: if {\tt n} is zero or negative, the loop never runs.
Otherwise, {\tt n} gets smaller each time through the
loop, so eventually we have to get to 0.

对于 \li{countdown} 来说,我们可以证明循环是一定会终止的:当 n 是 0 或者负数,该循环就不会执行;不然 n 通过每次循环之后慢慢减小,最终也是会变成 0 的。

For some other loops, it is not so easy to tell.  For example:

有些其他循环,可能就没那么好理解了。例如:

\begin{lstlisting}
def sequence(n):
    while n != 1:
        print(n)
        if n % 2 == 0:        # n is even
            n = n / 2
        else:                 # n is odd
            n = n*3 + 1
\end{lstlisting}

%
The condition for this loop is {\tt n != 1}, so the loop will continue
until {\tt n} is {\tt 1}, which makes the condition false.

循环的条件是 \li{n != 1},所以循环会一直执行到 \li{n} 等于 \li{1},条件判断为假时循环才终止。

Each time through the loop, the program outputs the value of {\tt n}
and then checks whether it is even or odd.  If it is even, {\tt n} is
divided by 2.  If it is odd, the value of {\tt n} is replaced with
{\tt n*3 + 1}. For example, if the argument passed to {\tt sequence}
is 3, the resulting values of {\tt n} are 3, 10, 5, 16, 8, 4, 2, 1.

每次循环,该程序打印出 \li{n} 的值,然后检查它是偶数还是奇数。如果它是偶数,
那么 \li{n} 可以被2整除;如果是奇数,则它的值被替换为 \li {n*3 + 1}。 例如,如果传递给 \li{sequence} 的实参为3, 那么打印出的结果将会是:\li{3}、 \li{10}、 \li{5}、 \li{16}、 \li{8}、 \li{4}、 \li{2}、 \li{1}。

Since {\tt n} sometimes increases and sometimes decreases, there is no
obvious proof that {\tt n} will ever reach 1, or that the program
terminates.  For some particular values of {\tt n}, we can prove
termination.  For example, if the starting value is a power of two,
{\tt n} will be even every time through the loop
until it reaches 1. The previous example ends with such a sequence,
starting with 16.
\index{Collatz conjecture}

由于 \li{n} 的值时增时减,所以不能轻易保证 \li{n} 会最终变成 \li{1}, 或者说这个程序能够终止。 对于某些特殊的 \li{n} 的值,可以很好地证明它是可以终止的。 例如, 当 \li{n} 的初始值是 \li{2} 的倍数时,则每次循环后 \li{n} 一直为偶数, 直到最终变为 \li{1}。 上一个示例中,程序就打印了类似的序列, 从 \li{16} 开始全部为偶数。
\index{Collatz conjecture}

The hard question is whether we can prove that this program terminates
for {\em all} positive values of {\tt n}.  So far, no one has
been able to prove it {\em or} disprove it!  (See
  \url{http://en.wikipedia.org/wiki/Collatz_conjecture}.)

难点在于是否能证明程序对于 {\bf 所有} 的正整数 \li{n} 都会终止。 目前为止,
还没有人证明 {\bf 或者} 证伪该命题。(见:\url{http://en.wikipedia.org/wiki/Collatz_conjecture} 。)

As an exercise, rewrite the function \verb"print_n" from
Section~\ref{recursion} using iteration instead of recursion.

我们做个练习,利用迭代而非递归,重写之前~\ref{recursion} 节中的 \li{print_n} 函数。


\section{{\tt break} }
\index{break statement}  \index{statement!break}

Sometimes you don't know it's time to end a loop until you get half
way through the body.  In that case you can use the {\tt break}
statement to jump out of the loop.

有些时候循环执行到一半你才知道循环该结束了。这种情况下,你可以使用 \li{break} 语句来跳出循环。

For example, suppose you want to take input from the user until they
type {\tt done}.  You could write:

例如,假设你想从用户那里获取输入,直到用户键入 \li{'done'}。 你可以这么写:


\begin{lstlisting}
while True:
    line = input('> ')
    if line == 'done':
        break
    print(line)

print('Done!')
\end{lstlisting}

%
The loop condition is {\tt True}, which is always true, so the
loop runs until it hits the break statement.

循环条件是 \li{True},其总是为真,所以该循环会一直执行直到碰到 \li{break}。

Each time through, it prompts the user with an angle bracket.
If the user types {\tt done}, the {\tt break} statement exits
the loop.  Otherwise the program echoes whatever the user types
and goes back to the top of the loop.  Here's a sample run:

每次循环时,程序都会给出一个尖括号 (\li{>}) 提示。 如果用户输入 \li{'done'},执行 \li{break} 语句跳出循环。 否则, 程序就会一直打印出用户所输入的内容并且跳到循环开始, 0以下是一个运行示例:


\begin{lstlisting}
> not done
not done
> done
Done!
\end{lstlisting}

%
This way of writing {\tt while} loops is common because you
can check the condition anywhere in the loop (not just at the
top) and you can express the stop condition affirmatively
(``stop when this happens'') rather than negatively (``keep going
until that happens'').

\li{while} 循环的这种写法很常见, 因为你可以在循环的任何地方判断条件
(而不只是在循环开始), 而且你可以积极地表达终止条件(``当出现这个情况是终止''), 而不是消极地表示 (``继续运行直到出现这个情况'')。



\section{Square roots  |  平方根}
\label{squareroot}
\index{square root}

Loops are often used in programs that compute
numerical results by starting with an approximate answer and
iteratively improving it.
\index{Newton's method}

循环常用于计算数值的程序中, 这类程序一般从一个大概的值开始, 然后迭代式地进行改进。 \index{Newton's method}

For example, one way of computing square roots is Newton's method.
Suppose that you want to know the square root of $a$.  If you start
with almost any estimate, $x$, you can compute a better
estimate with the following formula:

例如,牛顿法 (Newton's method) 是计算平方根的一种方法。 假设你想求 $a$ 的平方根。 如果你从任意一个估算值 $x$ 开始, 则可以利用下面的公式计算出更为较为精确的估算值:

\[ y = \frac{x + a/x}{2} \]

%
For example, if $a$ is 4 and $x$ is 3:

例如,假定 $a$ 是 4,$x$ 是 3:


\begin{lstlisting}
>>> a = 4
>>> x = 3
>>> y = (x + a/x) / 2
>>> y
2.16666666667
\end{lstlisting}

%
The result is closer to the correct answer ($\sqrt{4} = 2$).  If we
repeat the process with the new estimate, it gets even closer:

可以看到, 结果与真实值 ( $\sqrt{4} = 2$ ) 已经很接近了, 如果我们用这个值
再重新运算一遍, 它将得到更为接近的值。


\begin{lstlisting}
>>> x = y
>>> y = (x + a/x) / 2
>>> y
2.00641025641
\end{lstlisting}

%
After a few more updates, the estimate is almost exact:
\index{update}

再通过多几次的运算,这个估算可以说已经是很精确了。
\index{update}

\begin{lstlisting}
>>> x = y
>>> y = (x + a/x) / 2
>>> y
2.00001024003
>>> x = y
>>> y = (x + a/x) / 2
>>> y
2.00000000003
\end{lstlisting}

%
In general we don't know ahead of time how many steps it takes
to get to the right answer, but we know when we get there
because the estimate
stops changing:

一般来说, 我们事先不知道要多少步才能得到正确答案, 但是我们知道当估算值不再变动时, 我们就获得了正确的答案。


\begin{lstlisting}
>>> x = y
>>> y = (x + a/x) / 2
>>> y
2.0
>>> x = y
>>> y = (x + a/x) / 2
>>> y
2.0
\end{lstlisting}

%
When {\tt y == x}, we can stop.  Here is a loop that starts
with an initial estimate, {\tt x}, and improves it until it
stops changing:

当 \li{y == x} 时,我们可以停止计算了。下面这个循环就是利用一个初始估值 \li{x},
循序渐进地计算,直到估值不再变化。

\begin{lstlisting}
while True:
    print(x)
    y = (x + a/x) / 2
    if y == x:
        break
    x = y
\end{lstlisting}

%
For most values of {\tt a} this works fine, but in general it is
dangerous to test {\tt float} equality.
Floating-point values are only approximately right:
most rational numbers, like $1/3$, and irrational numbers, like
$\sqrt{2}$, can't be represented exactly with a {\tt float}.
\index{floating-point}  \index{epsilon}

对于大部分 \li{a} 的值,这个程序运行正常,不过一般来说,检查两个浮点数是否相等比较危险。浮点数只能大约表示:大多数有理数,如 $1/3$,以及无理数,
如 : $\sqrt{2}$,是不能用浮点数 ( \li{float} ) 来精确表示的。
\index{floating-point}  \index{epsilon}

Rather than checking whether {\tt x} and {\tt y} are exactly equal, it
is safer to use the built-in function {\tt abs} to compute the
absolute value, or magnitude, of the difference between them:

与其检查 \li{x} 和 \li{y} 的值是否完全相等,使用内置函数 \li{abs} 来计算二者之差的绝对值或数量级更为安全:

\begin{lstlisting}
    if abs(y-x) < epsilon:
        break
\end{lstlisting}

%
Where \verb"epsilon" has a value like {\tt 0.0000001} that
determines how close is close enough.

这里,变量 \li{epsilon} 是一个决定其精确度的值,如 \li{0.0000001}。


\section{Algorithms  |  算法}
\index{algorithm}  \index{算法}

Newton's method is an example of an {\bf algorithm}: it is a
mechanical process for solving a category of problems (in this
case, computing square roots).

牛顿法就是一个 {\em 算法} (Algorithm) 示例: 它是解决一类问题的计算机制
(本例中是计算平方根)。

To understand what an algorithm is, it might help to start with
something that is not an algorithm.  When you learned to multiply
single-digit numbers, you probably memorized the multiplication table.
In effect, you memorized 100 specific solutions.  That kind of
knowledge is not algorithmic.

为了理解算法是什么,先了解什么不是算法或许有点帮助。 你在学习一位数乘法时,
可能背出了乘法表。 实际上,你只是记住了 100 个确切的答案。 这种知识并{\bf 不是}算法性的。

But if you were ``lazy'', you might have learned a few
tricks.  For example, to find the product of $n$ and 9, you can
write $n-1$ as the first digit and $10-n$ as the second
digit.  This trick is a general solution for multiplying any
single-digit number by 9.  That's an algorithm!
\index{addition with carrying}  \index{carrying, addition with}
\index{subtraction!with borrowing}  \index{borrowing, subtraction with}

不过, 如果你想找 ``懒人方法'', 你可能就会找到一些诀窍。 比如为了计算 $n$
和 $9$ 的乘积,你可以把 $n-1$ 作为乘积的第一位数,再把 $10-n$
作为第二位数,从而得到它们的乘积。 这个诀窍是将任意个位数
与 $9$ 相乘的普遍解法。 这就{\bf 是}一种算法。
\index{addition with carrying}  \index{carrying, addition with}
\index{subtraction!with borrowing}  \index{borrowing, subtraction with}

Similarly, the techniques you learned for addition with carrying,
subtraction with borrowing, and long division are all algorithms.  One
of the characteristics of algorithms is that they do not require any
intelligence to carry out.  They are mechanical processes where
each step follows from the last according to a simple set of rules.

类似地,你所学过的进位加法、借位减法、以及长除法都是算法。算法的特点之一
就是不需要过多的脑力计算。算法是一个机械的过程,每一步都是依
据一组简单的规则跟着上一步来执行的。

Executing algorithms is boring, but designing them is interesting,
intellectually challenging, and a central part of computer science.

执行算法的过程是很乏味的,但是设计算法就比较有趣了,不但是智
力上的挑战,更是计算机科学的核心。

Some of the things that people do naturally, without difficulty or
conscious thought, are the hardest to express algorithmically.
Understanding natural language is a good example.  We all do it, but
so far no one has been able to explain {\em how} we do it, at least
not in the form of an algorithm.

一些人们自然而然无需下意识做到的事情,往往是难于用算法表达。 理解自然语言就是这样的。 我们每个人都听得懂自然语言, 但是目前还没有人能够解释我们是 {\bf 怎么} 做到的, 至少无法以算法的形式解释。

\section{Debugging  |  调试}
\label{bisectbug}

As you start writing bigger programs, you might find yourself
spending more time debugging.  More code means more chances to
make an error and more places for bugs to hide.
\index{debugging!by bisection}  \index{bisection, debugging by}

当你开始写更为复杂的程序时,你会发现大部分时间都花费在调试上。 更多的
代码意味着更高的出错概率,并且会有更多隐藏 bug 的地方。
\index{debugging!by bisection}  \index{bisection, debugging by}

One way to cut your debugging time is ``debugging by bisection''.
For example, if there are 100 lines in your program and you
check them one at a time, it would take 100 steps.

减少调试时间的一个方法就是“对分调试”。例如,如果程序有100行,你一次检查一行,就需要100步。

Instead, try to break the problem in half.  Look at the middle
of the program, or near it, for an intermediate value you
can check.  Add a {\tt print} statement (or something else
that has a verifiable effect) and run the program.

相反, 试着将问题拆为两半。 在代码中间部分或者附近的地方, 寻找一个可以检查的中间值。 加上一行 \li{print} 语句 (或是其他具有可验证效果的代码), 然后运行程序。

If the mid-point check is incorrect, there must be a problem in the
first half of the program.  If it is correct, the problem is
in the second half.

如果中间点检查出错了, 那么就说明程序的前半部分存在问题。 如果没问题, 则说明是后半部分出错了。

Every time you perform a check like this, you halve the number of
lines you have to search.  After six steps (which is fewer than 100),
you would be down to one or two lines of code, at least in theory.

每次你都这样检查, 就可以将需要搜索的代码行数减少一半。 经过6步之后(这比100小多了), 你将会找到那或者两行出错的代码, 至少理论上是这样。

In practice it is not always clear what
the ``middle of the program'' is and not always possible to
check it.  It doesn't make sense to count lines and find the
exact midpoint.  Instead, think about places
in the program where there might be errors and places where it
is easy to put a check.  Then choose a spot where you
think the chances are about the same that the bug is before
or after the check.

在实践中, 可能并不能很好的确定程序的 ``中间部分'' 是什么, 也有可能并不是那么好检查。
计算行数并且取其中间行是没有意义的。 相反, 多考虑下程序中哪些地方比较容易出问题, 或者哪些地方比较容易进行检查。 然后选定一个检查点, 在这个断点前后出现bug的概念差不多。


\section{Glossary  |  术语表}

\begin{description}

\item[reassignment:] Assigning a new value to a variable that
already exists.
\index{reassignment}

\item[重新赋值(reassignment):] 给已经存在的变量赋一个新的值。
\index{reassignment}

\item[update:] An assignment where the new value of the variable
depends on the old.
\index{update}

\item[更新(update):] 变量的新值取决于旧值的一种赋值方法。
\index{update}

\item[initialization:] An assignment that gives an initial value to
a variable that will be updated.
\index{initialization!variable}

\item[初始化(initialize):] 给后面将要更新的变量一个初始值的一种赋值方法。
\index{initialization!variable}

\item[increment:] An update that increases the value of a variable
(often by one).
\index{increment}

\item[递增(increment):] 通过增加变量的值的方式更新变量(通常是加 1)。
\index{increment}

\item[decrement:] An update that decreases the value of a variable.
\index{decrement}

\item[递减(decrement):] 通过减少变量的值的方式来更新变量。
\index{decrement}

\item[iteration:] Repeated execution of a set of statements using
either a recursive function call or a loop.
\index{iteration}

\item[迭代(iteration):] 利用递归或者循环的方式来重复执行代一组语句的过程。
\index{iteration}

\item[infinite loop:] A loop in which the terminating condition is
never satisfied.
\index{infinite loop}

\item[无限循环(infinite loop):] 无法满足终止条件的循环。
\index{infinite loop}

\item[algorithm:]  A general process for solving a category of
problems.
\index{algorithm}

\item[算法(algorithm):] 解决一类问题的通用过程。
\index{algorithm}

\end{description}


\section{Exercises  |  练习}

\begin{exercise}
\index{algorithm!square root}

Copy the loop from Section~\ref{squareroot}
and encapsulate it in a function called
\verb"mysqrt" that takes {\tt a} as a parameter, chooses a
reasonable value of {\tt x}, and returns an estimate of the square
root of {\tt a}.  \index{encapsulation}

复制 \ref{squareroot} 节中的循环, 将其封装进一个叫 {\em \li{mysqrt}} 的函数中。 这个函数接受 {\em \li{a}} 作为形参,选择一个合适的 {\em \li{x}} 值,并返回 {\em \li{a}} 的平方根估算值。

To test it, write a function named \verb"test_square_root"
that prints a table like this:

为测试上面的函数,编写一个名为 {\em \li{test_squre_root}} 的函数,打印出如下表格:

\begin{em}
\begin{lstlisting}
a   mysqrt(a)     math.sqrt(a)  diff
-   ---------     ------------  ----
1.0 1.0           1.0           0.0
2.0 1.41421356237 1.41421356237 2.22044604925e-16
3.0 1.73205080757 1.73205080757 0.0
4.0 2.0           2.0           0.0
5.0 2.2360679775  2.2360679775  0.0
6.0 2.44948974278 2.44948974278 0.0
7.0 2.64575131106 2.64575131106 0.0
8.0 2.82842712475 2.82842712475 4.4408920985e-16
9.0 3.0           3.0           0.0
\end{lstlisting}
\end{em}

%
The first column is a number, $a$; the second column is the square
root of $a$ computed with \verb"mysqrt"; the third column is the
square root computed by {\tt math.sqrt}; the fourth column is the
absolute value of the difference between the two estimates.

其中第一列是 $a$ 的值;第二列是通过 {\em \li{mysqrt}} 计算得到的 $a$ 的平方根; 第三列是用 {\em \li{math.sqrt}} 计算得到的平方根; 第四列则是这两个平方根之差的绝对值。


\end{exercise}


\begin{exercise}
\index{eval function}  \index{function!eval}

The built-in function {\tt eval} takes a string and evaluates
it using the Python interpreter.  For example:

内置函数 {\em \li{eval}} 接受一个字符串,并使用 Python 解释器来计算该字符串。例如:

\begin{lstlisting}
>>> eval('1 + 2 * 3')
7
>>> import math
>>> eval('math.sqrt(5)')
2.2360679774997898
>>> eval('type(math.pi)')
<class 'float'>
\end{lstlisting}

%
Write a function called \verb"eval_loop" that iteratively
prompts the user, takes the resulting input and evaluates
it using {\tt eval}, and prints the result.

编写一个名为 {\em \li{eval_loop}} 的函数,迭代式地提示用户输入, 获取输入的内容,并利用 {\em \li{eval}} 来计算其值,最后打印该值。

It should continue until the user enters \verb"'done'", and then
return the value of the last expression it evaluated.

该程序应持续运行, 知道用户输入 {\em \li{'done'}}, 然后返回它最后一次计算的表达式的值。

\end{exercise}


\begin{exercise}
\index{Ramanujan, Srinivasa}

The mathematician Srinivasa Ramanujan found an
infinite series
that can be used to generate a numerical
approximation of $1 / \pi$:
\index{pi}

数学家斯里尼瓦瑟·拉马努金 (Srinivasa Ramanujan) 发现了一个可以用来生成 $1 / \pi$
近似值的无穷级数 (infinite series):

\[ \frac{1}{\pi} = \frac{2\sqrt{2}}{9801}
\sum^\infty_{k=0} \frac{(4k)!(1103+26390k)}{(k!)^4 396^{4k}} \]

Write a function called \verb"estimate_pi" that uses this formula
to compute and return an estimate of $\pi$.  It should use a {\tt while}
loop to compute terms of the summation until the last term is
smaller than {\tt 1e-15} (which is Python notation for $10^{-15}$).
You can check the result by comparing it to {\tt math.pi}.

编写一个名为 {\em \li{estimate_pi}} 的函数,利用上面公式来估算并返回 $\pi$
的值。 这个函数应该使用 \li{while} 循环来计算所有项的和, 直到最后一项小于 {\em \li{1e-15}} (Python 中用于表达 $10^{-15}$ 的写法) 时终止循环。 你可以将该值与 $math.pi$ 进行比较, 检测是否准确。

Solution: \url{http://thinkpython2.com/code/pi.py}.

\href{http://thinkpython2.com/code/pi.py}{参考答案}

\end{exercise}
